%! TEX root = ../main.tex
%********************************************************************
% Appendix
% If problems with the headers: get headings in appendix etc. right
%\markboth{\spacedlowsmallcaps{Appendix}}{\spacedlowsmallcaps{Appendix}}

\chapter{Vortrag über das Epos (BSLS)}
Im Rahmen der zwölften jährlichen Tagung der \emph{British Society for
Literatur and Science} (BSLS) wurde folgender Vortrag im Rahmen eines 
  gemeinsamen Panels mit dem Thema \emph{Whither and whence Humanity} 
  gemeinsam mit den beiden Doktoranden des \emph{Erlanger Zentrums fÜr
  Literatur- und Naturwissenschaft} (ELINAS) Maria Wolff und Stefan Winter.
Der Vortrag hatte den Titel \emph{First~Earth --- Raoul Schrott’s New Epic
  and Contempory Physics}.
\graffito{Die Tagung fand vom xx.3.17 bis zum xx.3.17 in Bristol, England, statt.}

\section{Abstract}
In late 2016, Austrian writer, translator, and comparative literature
  scholar Raoul Schrott published an epic with the title Erste Erde. Epos. 
He combines creation myths, current scientific insights, and fictional life
  stories of scientists in the nowadays highly unusual and thus highly
  interesting form of an epic that exhibits many heterogeneous but playfully
  interwoven lyrical styles. 
He had been collaborating with scientists for several years in order to tell
  a creation story of our Earth starting with the Big Bang up to the
  “invention” of writing as scientifically correct as possible, while still
  being as poetic as possible.
In my talk I try to comment on and criticize this undertaking both from the
  perspective of a physicist working in a related field and from the
  perspective of a literary scholar.
I will center my investigations on how such an old poetic form as an epic
  might at the same time be one of the “most contemporary examples” of
  contemporary literature.
In order to do so, I will be concentrating on the first book of the epic
  which is mainly concerned with connecting today’s physics of whence our
  universe and in particular our Earth stems from with creation myths told
  by the Maori.

\section{Vortragstext}
For a very long time in the history of German literature---and to my best
  knowledge of World literature--- there has not been published a text of
  this genre and when it happened before it was a very rare occurence.
The genre I am talking about is the genre of an epic and in particular
  an epic transporting ideas on the genesis of the world.
But this time the effort has been made to not only create or present
  mythological world explanations: the task has been to write a scientificaly
  correct---as good as this is possible and as time depended as it is---epic
  tale.
The Austrian writer, translator, comparative literature scholar Raoul Schrott
  once translated one of the most famous epics that is out there: the iliad
  and now after having travelled and having gathered material for seven years
  his own epic \glqq Erste Erde\grqq\ (First Earth).
His goal has been at least as challangig as the book got thick!
