%! TEX root = ../main.tex
%************************************************
\chapter*{Vorbemerkungen}\label{ch:introduction}

Man nehme es mir nicht übel, dass die folgenden einleitenden Gedanken derart
  subjektiv gehalten sind, da es sich ja eigentlich um eine wissenschaftliche Arbeit
  handelt.
Nichtsdestotrotz wäre es wesentlich unwissenschaftlicher, die zutiefst
  subjektive Motivation dem aus ihr resultierendem Text nicht voranzustellen
  und so über die jeden Absatz, jede Zeile, jedes Wort durchdringenden
  Beweggründe den Schleier des Objektiven zu kleiden.
Die Grundahnung, die dieser Arbeit unterliegt, ist mir --~nach bestem Wissen
  und Gewissen~-- in gleichem Maße eine Herzens- wie Kopfangelegenheit (wenn
  man denn überhaupt platt in diese beiden Kategorien scheiden mag)
  und betrifft einen wesentlich meine Persönlichkeit bildenen 
  Aspekt, der fast meinen ganzen Lebensweg mitbestimmte --~wenn nicht leitete.

Seit ich ein kleines Kind war, überkommt mich von Zeit zu Zeit eine ganz
  innige Empfindung; eine Empfindung, die früh nur in einer schwachen Form 
  mit der Rezeption lieber aber gleichzeitig ernster Worte oder einer
  melancholischen und zugleich irgendwie doch euphorischen Melodie --~manchmal
  auch nur eines einzelnen Klanges in der Schwebe~-- verknüpft war;
  die aber heute genauso in Momenten des Nachdenkens über und des
  --~sicherlich selteneren, dann aber umso tieferen~-- Verstehens einiger
  Zusammenhänge in der Mathematik und Physik (oder generell in den
  Naturwissenschaften und der Philosophie) wie beim Wahrnehmen eines Gedichts
  oder von durchkomponierter Musik, beim Spielen des Klaviers oder Betrachten
  bildender Kunst aufkommen kann.
Sie ist ganz eng mit dem Rätsel der bewussten, qualitativen Wahrnehmung an sich
  verbunden, mit der Willensfreiheit, damit mit der Ethik und schließliche mit
  den Denkbewegungen des Erkennens und dann den Brüchen, Wendung und dem
  Anhalten darin: den Momenten der Erkenntnis; kurz: mit dem, was für mich
  wesentlich das Menschsein ausmacht.
In diesen Momenten kommt mir jegliches Verständnis für Lebensmüdigkeit
  abhanden und ich kann nicht verstehen, wie man nicht unfassbar überwältigt,
  beeindruckt von, in gewissem Sinne \emph{liebend} \so{allem} gegenüber sein
  kann und sich nicht ewig daran freuen kann.
Nicht platt sondern ganz wesentlich gemeint sind es diese Momente, nach welchen
  --~zumindest, denn mehr kann ich nicht wissen,~-- mich Mephistopheles in
  Fesseln schlagen dürfte.

In meiner Facharbeit im Leistungskurs Physik ging es \ua\ darum, die genaue
  Kurve zu berechnen, entlang derer eine Kette hängt. Man könnte meinen, es
  sei einfach eine Parabel; dem ist aber nicht so! Das Ergebnis ist seit
  langer Zeit bekannt und mir ging es nur darum, ohne nachzusehen, was die
  Lösung ist, ob ich sie selbst errechnen kann (im Rahmen der Schulmathematik)
  und danach auch überprüfen, ob die Kette wirklich entlang dieser Kurve hängt.
Es mag ein ganz einfaches Beispiel sein und keine der großen oder ganz tiefen
Erkenntnisse der Physik betreffen, aber es ist \emph{mein} Beispiel:
Man kann ganz elementar bei ersten Prinzipien anfangen, nämlich in diesem Fall
  bei den Newtonschen Gesetzen und dann nur durch saubere Logik und Mathematik
  aus ihnen herauspräparieren, dass die Kette nicht wie eine Parabel sondern
  zumindest sehr ähnlich wie der Graph einer sogenannten
  Kosinushyperbolicus-Funktion durchhängt.
Ich berechnete die Funktion dann genau für eine Kette einer bestimmten Länge,
  druckte den Graphen aus, stellte das Papier senkrecht auf, hing eine echte
  Kette über die ausgedruckte Linie und sah --~und in diesem Moment stellte
  sich das beschriebene Gefühl ganz besonders ein~-- wie der Verlauf der
  echten Kette mit dem Produkt theoretischer Überlegungen nahezu perfekt
  übereinstimmte.
  (Eine Vergleichsparabel gleicher Länge, von der die Kette merkbar abwich, war
  direkt auf das gleiche Blatt gedruckt.)
Mir lief buchstäblich ein wohliger Schauer über den Rücken. Nicht
  ausschließlich aber wesentlich auch staunend darüber, was der menschliche
  Geist vermag.
Er kann scheinbar vorher wissen, wie die Dinge sich verhalten werden und aus
  zu ganz weiten Teilen rein Theoretischem materielles Verhalten ableiten;
  nicht nur im Sinne etwas Erwartbarem, sondern auch in einer dem konkreten
  Menschen gänzlich neuen Zusammenhang.

Natürlich basieren diese Ableitungen auf etwas, das sich \ua\ aus
  experimentellen Daten gebildet hatte.
Die Geschichte der Naturwissenschaften lässt keinen Zweifel darüber, dass
  die Newtonschen Gesetze nicht das ausschließliche Produkt rein theoretischer
  Überlegungen sind, sondern dass sie wesentlich auf den ihnen vorausgehenden
  empirischen Erkenntnissen beruhen (nicht zuletzt auf Keplers Himmelsmechanik).
Dennoch konnte --~und kann~-- man sie aufstellen oder zumindest nachvollziehen,
  ohne irgendetwas mit beispielsweise Ketten ausprobiert zu haben; man kann
  sie aus theoretischen Prinzipien und basierend auf der Empirie der
  Planetenbewegungen oder der Empirie eines fallenden Apfels aufstellen.
Und dennoch erweist sich das aus ihnen Ableitbare in völlig anderen Bereichen
  der uns umgebenden Welt (eben \zB\ im Fall der Kette) als gewisse Macht des
  Geistes, völlig neue --~sich dann als (zumindest im Rahmen) richtige~--
  andere Erkenntnisse abzuleiten und damit wesentlich den Verlauf der Geschichte
  zu verändern; auch die der Entfaltung weiterer solcher Erkenntnisse.
Den ganzen Überlegungs- und Herleitungsprozess begleitend hat sich das Gefühl
  schon erahnen und immer stärker spühren lassen, bis es dann schließlich im
  Moment des Vergegenwärtigens der Übereinstimmung der Theorie mit dem Realen
  kulminierte und sich voll ausprägte, als noch eine weitere Zutate sich
  beimischte:
Welche Kette ich nehme, wann ich mich \emph{entscheide} das zu tun, welches
  Ende ich links welches rechts hinhänge, all das konnten --~und können~-- die
  Gleichungen der Physik nicht errechnen, aber wenn ich dann die Kette, die
  Länge, die Art der Aufhängung usw. \emph{gewählt} habe, hängt sie jedesmal
  genauso wie sich sehr genau errechnen lässt.
Der menschliche Geist in seinem Streben kann also gleichzeitg so wunderbar
  viel und so wenig.
Er scheint so \emph{mächtig} zu sein und gleichzeitig so beschränkt; ganz viel
  in der Wirklichkeit scheint bestimmt und (damit?) bestimmbar zu sein und ganz
  viel ist unbestimmt und (damit?) auch nicht bestimmbar.
  Hier stellt sich eine \emph{Grenzerfahrung} ein; der Mensch nimmt wahr, zu was der
  Geist fähig ist und dass er die Grenze dessen immer weiter hinaus treiben
  kann, bei gleichzeitiger bleibender Beschränktheit.
Er ist gleichzeitig tief in die Natur mit ihren Bestimmtheiten eingebunden und
  doch über sie herausgehoben.
Das Gefühl, das sich einstellt, wenn man erahnen kann, wie mächtig und frei der
  menschliche Geist ist bei gleichzeitiger Beschränktheit und Demut vor dem,
  was noch außerhalb liegt, ist für mich die Wahrnehmung des Erhabenen in
  Bezug auf Erkenntnis.
Es fusst gleichzeitig im Analytischen wie im Ästhetisch-Ethischen.
Denn auch beim Wahrnehmen einer Beethoven Symphonie oder \zB\ des dritten
  Klavierkonzerts von Rachmaninnof stellt sich ein Gefühl ein, dass davor
  erstaunnt und demütig zurücktritt, was der menschliche Geist hier zu Wege
  gebracht hat, wie einmalig, wie kunstvoll, wie authentisch, wie organisch
  und komplex hier etwas ist und gleizeitg wird es von so vielen menschlichen
  Parametern dominiert; es basiert auf so viel erlernbarer Struktur, letzlich
  wenigen Tönen, die uns eben gerade gefallen, Modulationsmustern, die man an
  beiden Händen abzählen kann, ganz viel Symmetrie und Einfachheit, wenn ich
  es höre bin ich gleichzeitig gezwungen in eine bestimmte Richtung zu denken, zu
  fühlen und gleichzeitig kann ich erahnen, wie frei der Mensch ist.
Es ist \emph{in einem} ein analytisches, ethisches und ästhetisches
  Phänomen, das Freiheit und Beschränktheit, Macht und Kleinheit des
  menschlichen Geistes wahrnehmbar machen und so erahnen lässt, was den
  fühlenden, denkenden und entscheidenden Geist wesentlich ausmacht.

Es sind die Empfindungen der \emph{Tiefe} und des \emph{Erhabenen}, die für
  mich die beiden Domänen der Kunst und des analytischen Denkens (sowohl der
  Geistes- wie auch der Naturwissenschaften) auf eine für mich ganz natürliche Weise
  immer schon verbanden und ich so bis heute nur die methodische nicht aber eine
  Trennung in der Tiefe nachvollziehen kann.
Mir liegt am Herzen, dass dieses volle, ausschöpfende Wahrnehmen sich genauso
  in beiden Bereichen entfaltet.
Mir fällt kein passenderes Thema ein, als an einem konkreten Gegestand diese
  Ahnung zu erforschen und ebenso sind mir keine aktuelles Werke bekannt, die das
  Erhabene am und im Überlapp von Naturwissenschaft und Kunst, speziell von
  Literatur und Physik, nicht nur berührt sondern zum Gegenstand hat wie Raoul
  Schrotts Gedichtband \emph{Tropen} und sein jüngstes Werk das Epos \emph{Erste Erde}.
Auch bei diesem Werk handelt es sich um ein extrem subjektives Unterfangen,
  das ein ganz objektives Unterfangen darstellen und mit dem Menschen
  verbandeln möchte: Das naturwissenschaftliche Weltbild. Raoul Schrott wollte
  sich nach eigener Aussage, all dies wunderbare Wissen aneigenen und will von
  diesen Erkenntnissen, Einsichten und auch Gefühlen berichten. Für ihn war
  das Verstehen untrennbar von der Lyrisierung, vom Metaphorisierung und in
  Bilder Fassen verbunden. Ein kosmogenetisches Epos mit dem Anspruch an
  --~zumindest aktuell~-- naturwissenschaftliche Korrektheit ist ein
  fantastisches und selbst hehres Unterfangen.
In diesem Werk das Verhältnis von Ästetik, Ethik und Analytik zu untersuchen
  und dabei auf die versöhnende Empfindung des Erhabenen einzugehen, könnte
  mir nicht mehr am Herzen liegen. 
Schiller \citep[S. 4]{SchillerNaiveDichtung}. Hegel ...

Schiller formulierte in seiner kurzen Zusammenfassung einer eigenen
  Geistesbewegung, die von einer Lektüre der Kantschen \glqq Kritik der
  Urteilskraft\grqq\ ausging und in eine Dramentheorie mündete, \glqq Das
  Pathetischerhabene\grqq, dass die \glqq Vorstellung eines fremden Leidens,
  verbunden mit Affekt und mit dem Bewu\ss tseyn unsrer innern moralischen
  Freyheit,\grqq\ das \glqq Pathetischerhabene\grqq\ sei
  \citep[S. ? f.]{SchillerPathetischErhaben}

Eine Hommage an die Reichhaltigkeit der Erscheinungen, an die unmittelbare
  Erfahrbarkeit natur- und geisteswissenschaftlicher Erkenntnisse und genauso
  deren Untersuchungsgegestände ist Raoul Schrotts Epos \emph{Erste Erde}
  \citep{Schrott2016ErsteErde}.
Es steht --~ohne Raoul Schrott eine Intention unterstellen zu wollen~--
  sowohl in seiner Genese performativ wie in seiner Endgestalt resultativ~--
  entschieden gegen eine Entdinglichung der Phänomene. Raoul Schrott brauchte
  den Bezug zu den Dingen. Er bereiste über sieben Jahre die Erde und besuchte
  Schlüsseldinge, -körper, schauplätze \ldots der wissenschaftlichen
  Entfaltung der Erkenntnis über unseren Planeten, unsere Kultur, unser
  Sonnensystem und das Universum. Er brauchte den echten \emph{Gegen}stand,
   \glqq denn ohne Gegen fällt man hart auf sich selbst.\grqq\ \citep[S.
   60]{Han2016}

Gleichzeitig sucht sie neben aller Direktheit und aller auratischer Momente
  auch das Erhabene des Erkennens über die Welt einzufangen und zu vermitteln;
  und dass neben aller Unmittelbarkeitmachung der --~genauer: wohl meist
  Schrotts~-- Erfahrungen auch das Pathos adressiert wird scheint nicht Zufall
  schon Kalkulation hinter dieser Textproduktion zu sein.
Schiller bemerkte darüber hinaus an anderer Stelle (diesmal zusammen mit
  Goethe), dass das epische Gedicht \glqq eine gewisse sinnliche Breite
  forder[e]\grqq, dass es vom \glqq\texttt{au\ss er sich wirkenden} Menschen
  handelt und dass der tragische Text im Gegensatz dazu den \glqq\texttt{nach
  innen geführten} Menschen \grqq\ darstellt.
Schrotts Epos ist also in gewissem Sinne beides: Epos und Tragödie; kurz: ein
  Text, der ganz nach Manier der Gegenwartsliteratur mit Gattungen spielt und
  ihre klassischen Grenzen zwar kennt, sie aber auch wenn nötig sprengt, um
  eine ~-- nicht nur, aber wesentlich auch~-- pathetisch-erhabene Erfahrung
  eines auf den ersten Blick vielleicht ganz unmenschlichen (a-menschlichen?)
  Themas zu ermöglichen. 

\glqq Man muß sich hüten, diese Meinung deshalb zu kritieren, weil sie so
  schwer auszusprechen ist; das liegt an unserer Sprache\grqq\
  \citep{Schroedinger1935}. 
\glqq Dem Wortlaut nach beziehen sich alle Aussagen auf das anschauliche
  Modell. Die wertvollen Aussagen sind an ihm wenig anschaulich und seine
  anschaulichen Merkmale sind von geringem Wert\grqq\ \citep{Schroedinger1935}.
Deduktiv und Induktiv, metaphysisch und konkret, theoretisch und
  experimentell: \glqq Der philosophische Begriff des Schönen, um seine
  wahre Natur vorläufig wenigstens anzudeuten, muß die beiden [\ldots] Extremen
  in sich vermittelt enthalten, indem er die metaphysische 
  Allgemeinheit mit der Bestimmtheit realer Besonderheiten verinigt. Erst so ist
  er an und für sich in seiner Wahrheit gefaßt.\grqq\ \citep[S. 39]{Hegel1986}.

Das Vorhaben dieser Arbeit ist, der \emph{sprachlichen} Verfasstheit der Empfindung
  der Erhabenheit in Raoul Schrotts Epos \emph{Erste Erde} als einer der
  Facetten des Transports von Erfahrung aus der Domäne naturwissenschaftlichen
  Wissens in den Leser nachzugehen; in anderen Worten also: das Bestreben die
  Literarisierung einer Facette der \emph{Ersten Erde} als Erzählung (im
  benjaminschen Sinne), nämlich der Überlagerung von Ethik, Ästhetik und
  Naturwissen; das Gewebe der Wörter vorsichtig aufzutrennen, die Machtrausch
  und Demut verbinden.
