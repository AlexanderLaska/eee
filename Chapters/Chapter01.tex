%! TEX root = ../main.tex
%************************************************
\chapter*{Literarisierung naturwissenschaftlicher
  Erkenntnis}\label{ch:Literarisierung}
Schiller formulierte in seiner kurzen Zusammenfassung einer eigenen
  Geistesbewegung, die von einer Lektüre der Kantschen \glqq Kritik der
  Urteilskraft\grqq\ ausging und in eine Dramentheorie mündete, \glqq Das
  Pathetischerhabene\grqq, dass die \glqq Vorstellung eines fremden Leidens,
  verbunden mit Affekt und mit dem Bewu\ss tseyn unsrer innern moralischen
  Freyheit,\grqq\ das \glqq Pathetischerhabene\grqq\ ist.

Eine Hommage an die Reichhaltigkeit der Erscheinungen, an die unmittelbare
  Erfahrbarkeit natur- und geisteswissenschaftlicher Erkenntnisse und genauso
  deren Untersuchungsgegestände ist Raoul Schrotts Epos \emph{Erste Erde}
  \citep{Schrott2016ErsteErde}.
Es steht --~ohne Raoul Schrott eine Intention unterstellen zu wollen~--
  sowohl in seiner Genese performativ wie in seiner Endgestalt resultativ~--
  entschieden gegen eine Entdinglichung der Phänomene. Raoul Schrott brauchte
  den Bezug zu den Dingen. Er bereiste über sieben Jahre die Erde und besuchte
  Schlüsseldinge, -körper, schauplätze \ldots der wissenschaftlichen
  Entfaltung der Erkenntnis über unseren Planeten, unsere Kultur, unser
  Sonnensystem und das Universum. Er brauchte den echten \emph{Gegen}stand,
  \glqq denn ohne Gegen fällt man hart auf sich selbst.\grqq\ \citep[S. 60]{Han2016}

Haptik stammt vom altgriechischen \textgreek{ἁπτικός}, \emph{haptik\'os}
  (etwa \glqq zum Berühren/Fühlen/Anfassen geeignet\grqq\ oder auch \glqq fähig
  berührt/gefühl/angefasst zu werden\grqq), das selbst wiederum durch
  Derivation aus dem Verb \textgreek{ἅπτειν}, \emph{h\'aptein} (anheften, in
  Kontakt bringen etc.), hervorging und wurde vom deutschen Psychologen Max
  Dessoir Ende des 19. Jahrhunderts eingeführt, um den Tastsinn systematisch
  neben Optik und Akustik einzureihen.

Gleichzeitig sucht sie neben aller Direktheit und aller auratischer Momente
  auch das Erhabene des Erkennens über die Welt einzufangen und zu vermitteln;
  und dass neben aller Unmittelbarkeitmachung der --~genauer: wohl meist
  Schrotts~-- Erfahrungen auch das Pathos adressiert wird scheint nicht Zufall
  schon Kalkulation hinter dieser Textproduktion zu sein.
Schiller bemerkte darüber hinaus an anderer Stelle (diesmal zusammen mit
  Goethe), dass das epische Gedicht \glqq eine gewisse sinnliche Breite
  forder[e]\grqq, dass es vom \glqq\texttt{au\ss er sich wirkenden} Menschen
  handelt und dass der tragische Text im Gegensatz dazu den \glqq\texttt{nach
  innen geführten} Menschen \grqq\ darstellt.
Schrotts Epos ist also in gewissem Sinne beides: Epos und Tragödie; kurz: ein
  Text, der ganz nach Manier der Gegenwartsliteratur mit Gattungen spielt und
  ihre klassischen Grenzen zwar kennt, sie aber auch wenn nötig sprengt, um
  eine pathetisch erhabene Erfahrung eines auf den ersten Blick vielleicht
  ganz unmenschlichen (amenschlichen?) Themas zu ermöglichen. 

dBachtin? Gattungstheorie ...
Quellen!

\glqq Man muß sich hüten, diese Meinung deshalb zu kritieren, weil sie so
  schwer auszusprechen ist; das liegt an unserer Sprache\grqq\
  \citep{Schroedinger1935}. 
\glqq Dem Wortlaut nach beziehen sich alle Aussagen auf das anschauliche
  Modell. Die wertvollen Aussagen sind an ihm wenig anschaulich und seine
  anschaulichen Merkmale sind von geringem Wert\grqq\ \citep{Schroedinger1935}.

Deduktiv und Induktiv, metaphysisch und konkret, Theoretische und
  Experimental-Physik: \glqq Der philosophische Begriff des Schönen, um seine
  wahreNatur vorläufig wenigstens anzudeuten, muß die beiden [\ldots] Extremen
  in sich vermittelt enthalten, indem er die metaphysische 
  allgemeinheit mit der Bestimmtheit realer Besonderheiten verinigt. Erst so ist
  er an und für sich in siner Wahrheit gefaßt.\grqq\ \citep[S. 39]{Hegel1986}.


\subsection{Benjamin}
In seinem berühmten Erzähleraufsatz, den \emph{Betrachtungen zum Werk Nikolai
  Lesskows}, konstatierte Walter Benjamin schon 1936, dass \glqq es mit der
  Kunst des Erzählens zu Ende geht.\grqq\ \citep[S. 385]{Benjamin1936Erzaehler}
Er wusste noch nichts von der digitalen Revolution, den Computern, dem
  Internet; nichts von Twitter, Blogs und Youtube.
Er sah damals schon, dass solche Medien --~vermutlich hatte er vornehmlich
  Zeitungen im Sinn~-- im Wesentlichen keine \emph{Erfahrung} sondern
  \emph{Information} transportieren.
Information ermangle eine gewissen \glqq Schwingungsbreite\grqq\ und sie
  \glqq hat ihren Lohn mit dem Augenblick dahin, in dem sie neu
  war.\grqq\ \citep[beide S. 391]{Benjamin1936Erzaehler}
Das Wesentliche einer Erzählung ist also für Bnejamin, dass sie echte Erfahrung
  vermittelt und dafür muss der Erzähler die \glqq innigste Durchdringung [der]
  beiden archaischen Typen\grqq\ \citep[S. 386]{Benjamin1936Erzaehler}
  --~Seemann und Ackerbauer~-- realisieren, so wie \glqq das Mittelalter in
  seiner Handweksverfassung zustande \grqq\ \citep[ebd.]{Benjamin1936Erzaehler} brachte.
Raoul schrott scheint hier prädesteniert zu sein --~, wenn es nicht inszeniert
  ist~--, da genauso an einem Ort verfeilen und ihn ganz genau aufnehmen,
  Erfahrungen \emph{in der Heimat} machen, wie auf der ganzen Welt
  herumreisen und neue Erfahrungen sammeln kann.
Sein als Haptik-Fetisch anmutendes Recherche-Verhalten erscheint so in neuem
  Licht.
Auf der Suche nach echten Erfahrungen, muss er auch den Seefahrer verkörpern,
  der echte Kunde mit nach Hause bringt.
\glqq [I]n jeden Fall iser der Erzähler ein Mann, der dem Hörer Rat
  wei\ss\ [\ldots] [und] Rat, in den Stoff gelebten Lebens eingewoben, ist Weisheit.
  Die Kunst des Erzählens neigt sich ihrem Ende zu, da die epische Seite der
  Wahrheit, die Weisheit, ausstirbt. \grqq\ \citep[S. 388]{Benjamin1936Erzaehler}
Ob Schrott es übertreibt und zu gewollt seine Weisheiten der \emph{Ersten
  Erde} --~, um es mit benjaminscher Terminologie zu sagen,~-- einsenkte,
  ist eine durchaus interessante Unterstellung, die später im Text verhandelt
  werden soll (\ref{...}).
Im Rahmen des Erzähleraufsatzes lässt sich jedoch anmerken, dass dieser
  Eindruck vielelicht in Ermangelung einer Zurückhaltung entsteht, die auch
  Benjamin vorschwebte:
\glqq Es gibt nichts, was Geschichten dem Gedächtnis nachhaltiger anempfiehlt
  als jene keusche Gedrungenheit, welche sie psychologischer Analyse
  entzieht.\grqq\ \citep[S. 392]{Benjamin1936Erzaehler}

Warum entschied Raoul Schrott sich für die Gattung \emph{Epos} und nicht wie
  es auf den ersten Blick naheliegender scheinen könnte für einen
  (Bildungs-)Roman?
Benjamin sieht ihm Aufkommen des Romans schon die ersten Anzeichen für den
  \glqq Niedergang des Erzählens\grqq\ \citep[389]{Benjamin1936Erzaehler} und
  zieht eine kategorische Trennlinie zwischen Roman und Epik:
  \glqq Das mündlich Tradierbare, das Gut der Epik, ist von anderer
  Beschaffenheit als das, was den Bestand des Romans ausmacht.\grqq und
  weiter \glqq Der Erzähler nimmt, was er erzählt aus der Erfahrung; aus der
  eigenen oder berichteten. Und er macht es wiederum zur Erfahrung derer, die
  seiner Geschichte zuhören\grqq und schlie\ss lich \glqq Einen Roman schreiben
  hei\ss t, in der Darstellung des menschlichen Lebens das Inkomensurable auf
  die Spitze treiben.\grqq.
(Schrott scheint hier dem Genre Epos nicht ganz treu zu bleiben, denn \zB die
  Geschichte von XYZ, die in die Erste Erde eingewoben ist, hat durchaus etwas
  romanhaftes in diesem Sinne. Es ist durchaus psychologisch, innig, subjektiv
  und das Inkomensurable darstellend \ldots)
Benjamin kommt jedenfalls zur Einsicht, dass indem der Bildungsroman \glqq den
  gesellschaftlichen Lebensproze\ss\ in der Entwicklung einer Person integriert,
  [\ldots] er den ihn bestimmenden Ordnungen die denkbar brüchigste
  Rechtfertigung angedeihen\grqq\ \citep[ebd.]{Benjamin1936Erzaehler} lä\ss t.
Hier muss natürlich noch \glqq gesell-\grqq\ durch \glqq
  naturwissen-\grqq\ und \glqq Leben-\grqq durch \glqq Erfahrung-\grqq erstetzt
  werden, was aber der nachfolgenden Schlussfolgerung keinen Abruch tut.
Es mag Ausnahmen geben wie gerade eben Goethes \emph{Wahlverwandtschaften},
  in denen eine naturwissenschaftliche Ordnung, die einer 
