%! TEX root = ../main.tex
%************************************************
\chapter*{Abriss der Handlung}\label{ch:summary}

Eine Zusammenfassung der Handlung lässt sich schwer in der üblichen Form
  angeben, da es sich um viele Episoden handelt, die nicht einer durchgängigen
  Handlungentwicklung folgen.
Selbst die einzelnen Handlungsstränge sind sehr fragmentarisch und nicht
  zuletzt --~\dhei vielleicht am erschwerensten~-- ist der Umstand, dass die
  ganz viel erkländes naturwissenschaftliche, 
  kulturwissenschaftliches und geisteswissenschaftliches Wissen durch eher
  \glqq Minimalhandlungen\grqq\ gerahmt wird.
Nichtsdestotrotz ist der Text so umfassend, dass sich ein kurzer Abriss der
  verschiedenen über den Globus verteilten Orte, an denen erzählt wird, der
  auftretenden Personen und letztlich der naturwissenschaftliche Gehalt, die
  hier zur Handlung gemacht werden; ganz wesentlich anders als in fast aller
  anderen Literatur. 

Die schwangere Ahellegen Moore besichtigt mit ihrem Mann George Allan Moore
  die unterirdischen Glühwürmchen-Höhlen von Waitomo. Raoul Schrott erzählt
  durch sie den letzten großen Schöpfungsmythos, der von den M\-{a}ori stammt und
  Mitte des 19 Jahrhunderts sich erst in der präsentierten Form bildete.

Um zu entscheiden, ob sich auf dem Gipfel des Cerro Armazones sinnvoll das
  dann größte Spiegelteleskop der Welt bauen lässt, ist George mit den beiden
  Astronomen Nagayoshi Li aus Taiwan und Michael Höss aus Deutschland
  in der Atacama-Wüste.

Zu dritt verbringen sie Silvester 2009 in Chile auf dem Weg zum Observatorium in
  Paranal.
Dabei erzählt Nagayoshi von seiner Liebe zu einer chinesischen Frau vom
  Festland und webt dabei den Vorgang der Planetenentstehung ebenso wie
  typische Vorurteils- und Konfliktstrukturen, die ihm begegneten, ein.
Darüber hinaus lässt er einen pekischen Sternenwärter aus dem 11.
  Jahrhundert zu Wort kommen, der Supernovae beobachtet; Explosionen am Ende der
  Leben der Sterne --~ohne sie damals natürlich als solche konzipiert zu haben~-

Michael erzählt am Morgen des Neujahrtages von Meteoriten, ihrer Rolle bei der
  Entstehtung unseres Sonnsystems und aber auch ihrem symbolischen Moment.

Detlev Orlo erzählt von den verschiedensten physikalischen Prinzipien,
  Einsichten und Gesetzen, deren Realisierungen auf seinen Bildern zusehen sind,
  die auf einer Ausstellung in Essen zu bestaunen sind.

In der nächsten Episode erklärt Raoul Schrott selbst seiner Tochter die
  Planetenentstehung und die zum Teil harschen Kollisionsprozesse im jungen
  Sonnensystem, während sie Silvester 2010 den aktiven äthiopischen Vulkan
  Erta Alé besteigen.

Wieder ein Jahr später unternimmt die Schriftstellerin Martina Guiliani aus
  Berlin zusammen mit einem befreundeten Arzt und einem Inuit-Führer in Kanada
  eine Expedition zu einer Insel, auf der man das älteste noch erhalten
  gebliebene Gestein mit eigenen Händen anfassen kann.
Bei der Reise geraten sie in Stromschnellen und verlieren ihre Nahrungsmittel,
  ihr Kartenmaterial und eine Waffe; Sie erreichen auch den Treffpunkt nicht,
  von dem aus sie abgeholt worden wären.
Schließlich müssen sie mit einem Flugzeug gesucht werden, das sie aber auch
  findet und zurückbringt.
%
Der aus Schottland stammende Landschaftsarchitekt Carl Jenk, der am
  Teilchenphysik-Forschungsinstitut CERN einen kosmologischen Park angelegt hat
  ist das Sprachrohr, durch das hindurch die rauen Prozesse der jungen Erde
  und des Mondes bis hin zum letzten \glqq Großen Bombardement\grqq\ durch
  Kometen- und Asteroidenschauer.
Er hat nun auch um eine Kathedrale einen solchen Park angelegt und will die
  Endrücke der gewaltigen Einschläge in einem Triptychon mittels Glasmalerei 
  umsetzen.

Wieder Raoul Schrott selbst erzählt auf der Folie des Quellwassers der
  Hippokrene auf dem Helikon in Griechenland die Entstehung der ersten
  beochemischen Moleküle.
Die angelegte Parallele zwischen Dichtung und Leben, die hier beide als aus
  dem Wasser entstehend verortet werden, wird weitergesponnen und so \zB Worte
  und Zellen analogisiert.

Auf Island erzählt der einheimische Vulkanologe Einar Sigursson nicht nur
  einen orphischen Schöpfungsmythos, sondern anhand der Exponate seines
  Museums in Stykkisholmur auch viel über die Chemie der Geologie, die
  Entstehung der verschiedenen Gesteinssorten, dass unser Sonnensystem aus den
  Resten einer Sonnenexplosion entstanden sein muss und vom gegenseitigen
  aufeinander bezogen Sein der Minerale und des Lebens.

Bei einer seiner Exkursionen 2014 nimmt die holländische Chemikerin Karen Lender
  teil, die kurz vor einer ernsten Operation steht, nachdem bei ihr im
  Brustkrebs diagnostiziert wurde.
Die Theorien der Entstehung des Lebens in heißen Quellen am Meeresgrund wird
  ebenso wie Gedanken über die Chemie des Lebens und die Entstehung von Krebs
  an der Exkursion entfaltet.

Einar hebt den Phosphor heraus und berichtet an der Geschichte des Alchemisten
  und Forschers Henning Brands sowohl über die Bedeutung des Elementes als
  über seine Gewinnung und Entdeckung.

Die nächste Episode wird wieder von Raoul Schrott selbst vorgetragen.
Er bestaunt die Architektur der Stadt Batumis am schwarzen Meer in Georgien und besingt
  durch sie inspiert auch die Architektur des Lebens: Zuerst der Elemente aus
  den Bausteinen Protonen, Neutronen und Elektronen und wie sie ganze
  Periodensystem bilden und davon ausgehend die Moleküle bis hin zur RNA und
  DNA, die den Aufbau aller Lebewesen kodieren.

Auch der Erzähler des nächsten Abschnitts ist Raoul Schrott selbst, der
  diesmal zu den ältesten Fossilien reiste.
Dazu startete er auf den Kermadec Inseln (Neuseeland) und besucht über die
  Jack Hills kommend die Shark Bay Westaustraliens.
Über ein Lokalblatt erfährt Raoul Schrott --~und mit ihm der Leser~-- von den
  Schöpfungsvorstellungen der Aborigines.
Mit diesem Ort sind die naturwissenschaftlichen Vorstellungen über die
  Bedingungen ersten Lebens verbunden: \ZB die Entstehung der Luft in ihrer
  groben Zusammensetzung, wie sie auch heute noch unzähligen Lebewesen geatmet
  wird.

Das Wissen über das Aufkommen verschiedener Geschlechter, von Sex, Familie und
  Tod in der Natur wird in inneren Zwiegesprächen der pensionierten
  Mikrobiologin Marz McCallum mit ihrem versrobenen Mann vorgetragen.
Sie sitzt in einer Londoner U-Bahn-Station, um dort die Ansagestimme ihres
  Mannes hören zu können.
Sie entspinnt ihre naturwissenschaftliche Narration um die Kragengeissler,
deren Flimmerhärchen eine essentielle Rolle bei der Entstehung der Nerven, der
  Sinneszellen aber genauso auch der Eileiter und Spermien spielten.

Erneut egreift Raoul Schrott das Wort, aber diesmal zusammen mit dem diesmal
  nicht fiktiven deutschen Wissenschaftler Christian Gottfried Ehrenberg.
Gemeinsam wird erst über die Bioluminiszenz berichtet, dann einzeln von
  Ehrenberg selbst über seine Entdeckung des Echeria coli Bakterium in Ägypten
  und wieder zurück in Berlin von mikrosopischen Urlebewesen, sog. Protisten,
  und Algen; wie sie Teppiche bilden, sog. Meteorpapier.
Raoul Schrott rundet den Abschnitt durch Zeilen Meeresleuchten, vieler kleiner
  Forscherportraits, die thematisch Verwandtes beitrugen und zum \glqq Licht
  in allem\grqq\ ab.

In Schottland schwelgt er in der Bedeutung des Eises, dem Aufkommen der
  Eiszeiten, dem Zusammenspiel von Wasser, Eis und Sauerstoff und so auch der
  Entstehung der pro- und eukaryontischen Zellen.

Eine neue Gestalt tritt auf: der französische Augenarzt Yves Marengo, dessen
  Frau, eine deutsche Schauspielerin, 2013 Suizid begang.
Er verarbeitet seine Trauer, indem er eine fiktive Autopsie seiner toten frau
  schreibt, anhand der Raoul Schrott die Architektur unseres Körpers darlegt.
Yves befindet sich zuerst in Ägypten am Roten Meer, dann Frankreich und Kanada
  und letztlich in Marokko.
Bei jeder Station werden neue Facetten des menschlichen Körpers illustriert;
von den Gliedern, über die Muskeln, bis hinzu den Augen.

Raoul Schrott lässt erst in seiner jetzigen Wahlheit Bregenzerwald (Österreich)
  und dann auf einer Irland-Reise den Botaniker Thomas Amann anhand der
  Geschichte dessen letztlich an einer Totgeburt gescheiterter Ehe die
  Eroberung des Landes durch das Leben erzählen.
Gletscher und alpine Seen sind hierfür die Assoziationsquellen.

Erneut spricht der Autor selbst über eine Reise von den Vereinigten Staaten
  hinüber nach Kanada in den Miguasha National Park.
In einem zugehörigen Museum entfaltet sich an den Exponaten Wissen über die
  ersten Landlebewesen und Wirbeltiere, ihre Emanzipation von ihren maritimen
  Vorfahren und auch unsere Wurzeln bei diesen.

Wiederum spricht ein realer Wissenschaftler, der österreicherische Zoologe und
  Naturforscher Johann Natterer, der nach einer Expedition für Kaiser Franz
  den 1. in Brasilien hängenbleibt und für ein geplantes Museum dort Pfalnzen
  und Tiere des Amazonasgebiets sammelt.
Darwinsche Ideen und weiteres Wissen über die Entstehungsgeschichte mehrer
  Körperteile wie Elle und Speiche finden sich hier eingewoben.
Es kommen auch die Zoologen Leopold (Johann) Fitzinger, Wien, Richard
  Owen, London, Richard Swann von der Yale-Universität, New Haven, und Neil
  Jenkins aus Kanada zu Wort; bis die Zergliederung in Reptilien, Vögel und
  Säugetiere vorgetragen ist

Zofia Kalin-Halzska eine emeritierte Zoologie-Professorin aus Polen berichtet
  erst über Polen, dann Expedition in die Mongolei und Mexico.
Ihre Erzählung über das eigene Leben ist mit einer über das Leben im
  Allgemeinen verwoben: Ihr Fachgebiet ist \ua die Ausbildung der Milchdrųsen
  und das Aufkommen der Warmblüter.
Sie erzählt ihre teils grausame Geschichte einem Reporter.
Bedeutenden Ergeignisse wie der Besuch des Kraters, der wohl durch den Einschlag des
  Asteroiden, der die Saurier ausrottete, entstand, bis hin zu ihrer Vergewaltigung und
  Folter sind Teil ihrer Erzählung.
Ein Jahr nach dem Interview stirbt sie.

Nach dem Rückflug von Grönlad landet Raoul Schrott in Innsbruck und kleidet in
  eine Landschaftsbeschreibung von Tirol und Vorarlberg, seiner Heimat, die
  Erkenntnisse über die Geologie der Alpen im Speziellen und der \glqq
  Gestalungsformen der Erde\grqq\ im Allgemeinen.

Das Zusammenspiel der Verhaltensforscherin Anja Magall aus Deutschland und dem
  schweizer Konzervorstands Christopher Suddendorff, die sich zwar auch im
  Leipziger Zoo treffen, aber wegen der Distanz auch oft zwischen Zürich und Leipzig Briefe
  (oder E-Mails) schreiben, bildet die Folie auf der wissenschaftliche
  Vorstellungen über menschliche Verletzlichkeit genauso wie über Machtwillen,
  Gruppendynamik und Sexualverhalten eröffnet werden.

Raoul Schrott spannt im nächsten Abschnitt wieder selbst aber zusammen mit dem
  Paläontologen Fidelis Masao anhand der Expeditionen in Äthiopien und
  Tansania zu den ersten Hominiden-Ausgrabungsstätten den Bogen vom Aufkommen
  des aufrechten Ganges bis hinzu zu den ältesten Funden von Symbolen, die der
  Mensch produzierte.

Die Entwicklung des Homo erectus zum kulturellen \glqq Superorganismus\grqq\
  der Zivilisation wird von der amerikanischen Kunsthistorikerin Frances Wolf
  von Ausstellungsstücken in ihrem New Yorker \glqq Alternativen Museum\grqq\
  ausgehend erzählt.
Die Bilder stellen das dar, was nach einer Volksbefragung im statistischen
  Mittel am ehesten gesehen werden wollte.
\Ua werden auch Erkenntnisse über Frühmenschen in der Kaukasus-Region in ihre
  eigene Familiengeschichte integriert. So ist \zB ihre Mutter aus Georgien
  geflohen, wodurch der lokale Bezug zu den wissenschafltichen Inhalten
  hergestellt wird.

Grundfragen über die Entstehung der Schrift, die ersten Bilder und ihre
  vielleicht metaphysische, vielleicht reale Bedeutung werden in den
  Erzählungen Heinrich Siffers, einem deutschen Archäologen, reflektiert.
Er betreibt 2015 in den Höhlen in Deutschland, Frankreich, Spanien und Österreich
  nicht nur Archäologie, sondern auch die Untersuchungen über den Einfluss der
  absoluten Dunkelheit und Einsamkeit auf sein Körper- und zeitgefühl, nachdem
  kurz zuvor durch den Tod seiner Frau buchstäblich aus dem Gleichgewicht
  geriet.
Auch das Aufkommen von Kunst und Religion wird hier in den Höhlenmalerein verortet.

Das Buch beschließt mit den Worten Raoul Schrotts, der erneut daheim ist, um
  dort das Gefühl einer gewissen Distanz und Gleichgültigkeit der Heimat
  gegenüber zu bemerken.
Gleichzeitig drückt er aus, wie viel \zB in den Ortsbezeichnungen versteckt
  ist und gleichzeitig, wie unzureichend alle Bennenung scheint.
Das Epos endet so wieder im Namenlosen der unmittelbaren Erfahrung, die nach
  aller sozialen, ästhetischen und wissenschaftlichen Erfahrung der 650
  vorangehenden Seiten immer noch etwas Hartes, Fremdes und Authentisches
  behalten hat.
