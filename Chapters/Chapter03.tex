%! TEX root = ../main.tex
%************************************************
\chapter{Synthese}\label{ch:synthese} % $\mathbb{ZNR}$
%************************************************

\section{Die Penetranz des Homo-Mensura-Satzes}

Es ist, als ob die Erste Erde vom protagoreischen Homo-Mensura-Satz geradezu
  durchsetzt sei; also der Idee, der Mensch sei das Ma\ss\ aller Dinge.
Raoul Schrott lässt keine Gelegenheit aus, dem Text seine vermeintliche Einsicht
  aufzunötigen, dass --~wie er es vor dem eigentlichen Fließtext sogar schon im
  Inhaltsverzeichnis ausdrückt~-- die grundlegensten wissenschaftlichen
  Theorien von alten menschlichen Vorstellungen (\zB\ antiken Mythen) geprägt seien
  und \zB\ \glqq das Wissen über [beispielsweise] die Bildung von Galaxion 
  \emph{unsere} Denkweisen wiederspiegelt\grqq
  \footnote{Die Kursivierung im Zitat stammt nicht von Raoul Schrott.}
  \citep[S. 5]{Schrott2016ErsteErde}.
Er vergisst --~oder übergeht~-- zu oft das Wörtchen \glqq auch\grqq.
Das geht so wesentlich gegen das Erhabene der naturwissenschaftlichen
  Erkenntnis, um die es ihm eigentlich bestellt zu sein scheint. 
Besonders die theoretische Physik und Mathematik, aber im Kern jede
  Naturwissenschaft steht aber gerade dazu in Opposition.
Selbstverständlich wird sie den menschlichen Einfluss und von ihrem
  geschichtlichen Verlauf her allein schon in ihrer inspirativen Dimension
  nicht die mythischen Einflüsse los.
Aber gerade die beiden großen Säulen der Physik, die Allgemeine
  Relativitätstheorie, die Grundlage der in der Ersten Erde verhandelten
  physikalischen Kosmologien / -gonien, und die Quantenmechanik, mit der sich
  Schrott nachweislich schon lange beschäftigt (siehe \zB\ sein Gedichtband
  \emph{Tropen}), sind paradigmatisch für wissenschaftliche Einsichten, die
  aus guten physikalischen --~insbesondere empirischen~-- und aber genauso 
  theoretisch-philosophischen Gründen geradezu dadurch ausgezeichnet zu sein
  scheinen, in weiten Teilen von der Alltagserfahrung, dem gesunden
  Menschenverstand, mythischen Vorstellungen und menschlichen Begriffen zu
  differieren.
Ganz viel wissenschaftliche Feinarbeit ist sowohl von den ganz Großen des
  Fachs als auch von einem Meer an Forschern geleistet worden, um nachzuweisen,
  ob sich die Quantenmechanik letztlich doch mit intuitiveren, menschlicheren
  Vorstellungen fassen lasse.
Aber alle scheiterten und wie es (gleichzeitig
  theoretisch wie auch empirisch) aussieht auf eine ganz wesentliche und tiefe Weise
  (vom EPR-Paradoxon über die Bellschen Ungleichungen bis hin zum
  \mbox{Kochen-Specker-Theorem}).
Es ist eher so, dass die Physik versucht, auszuhandeln, wie die Dinge, die sie
  untersucht \emph{wirklich}, \dhei\ undabhängig davon, wie wir sie gerne hätten,
  damit sie leicht eingänglich sind, funktionieren; zumindest, wie wir sie beschreiben können
  und in jedem Fall hat sich herausgestellt, dass sich gerade je weiter die
  Wissenschaft fortschreitet (d.\,h. je mehr Phänomene sie erfolgreich in ihr
  Weltbild bei möglichst wenigen Annahmen eingliedern kann), desto ferner
  sich die Theoriegegenstände vom Menschen und unseren mythischen
  Vorstellungen entfernen zu scheinen.
Niemand weiß, ob sich nicht alles eines Tages in einem einfachen Bild
  zusammenfassen lässt und dieses aus welchen Gründen auch immer eine Nähe zu
  mythologischen Vorstellungen besitzt.
Nach bestem Wissen und Gewissen: Es sieht einfach nicht danach aus.
Genauso selbstverständlich haben wir Menschen dann die Tendenz, uns dennoch
  diese amenschlichen Umstände \glqq mit [unseren] menschlichen Maßstäben
  modellhaft anschaulich [zu] machen.\grqq, was Schrott aber
  selbst-inkonsistent"-erweise als nicht möglich abtut \citep[vgl. S. 5]{Schrott2016ErsteErde}. 
Er demonstriert dem zum Trotz weiter hinten im Text, wie gut dies geht; \ZB\
  als er die Abstands- und Größenverhältnisse im Sonnensystem mit von einem Auto fallendem
  Obst illustriert \citep[S. 48 f.]{Schrott2016ErsteErde}.


Angebrachter wäre, der Mensch ist \emph{ein} Ma\ss\ \emph{vieler} Dinge und für beide
  Ausprägungen (wie stark er Ma\ss\ des jeweiligen Dinges und für welche
  überhaupt) wiederum ein Ma\ss\ anzugeben, anstatt penetrant gerade den
  Errungenschaften des menschlichen Geistes, bei denen er am fairsten von sich
  abstand nimmt --~sich methodisch heraushällt~-- und Strukturen des
  Weltganzen offenlegt, die so ganz und gar nicht anthropomorph sind, zu
  unterstellen, sie seien wesentlich von menschlichen --~vor allem
  mythischen und insbesondere eben~-- fiktiven Gehalten durchsetzt.
Vielleicht ist es eher so, dass es eben eine Mannigfaltigkeit an menschlichen
  Vorstellungen, Geschichten und Mythen gibt, die auf \emph{manche}
  wissenschafltichen Ergebnisse \emph{teilweise} passen; In diesen Fällen
  empfehlen sich diese \glqq Geschichte\grqq\ dann natürlich dem Geist an, da
  sie of grundmenschle Erfahrungen verpacken und so der abstrakte Gehalt
  leichter vermittelt werden kann.
Aber dann kann man freilich auch ein Narrativ spinnen, in dem diese Vorstellungen,
  Geschichten und Mythen unseren heutigen wissenschaftlichen Vorstellungen vermeintlich
  wesentlich vorrausgehen und es so darstellen, als treten wir quasi auf der Stelle und
  seien nicht fähig wesentlich weiter über den Tellerrand hinauszusehen, als
  wir es in Homers Odyssee, Hesiods Theogonie oder Maori Mythen Waitomos taten.
