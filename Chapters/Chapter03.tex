%! TEX root = ../main.tex
%************************************************
\chapter{Synthese}\label{ch:synthese} % $\mathbb{ZNR}$
%************************************************

\section{Die Penetranz des Homo-Mensura-Satzes}

Es ist, als ob die Erste Erde vom protagoreischen Homo-Mensura-Satz geradezu
  durchsetzt sei; also der Idee, der Mensch sei das Ma\ss\ aller Dinge.
Raoul Schrott lässt keine Gelegenheit aus, dem Text seine vermeintliche Einsicht
  aufzunötigen, dass --~wie er es vor dem eigentlichen Fließtext sogar schon im
  Inhaltsverzeichnis ausdrückt~-- die grundlegensten wissenschaftlichen
  Theorien von alten menschlichen Vorstellungen (\zB\ antiken Mythen) geprägt seien
  und \zB\ \glqq das Wissen über [beispielsweise] die Bildung von Galaxion 
  \emph{unsere} Denkweisen wiederspiegelt\grqq
  \footnote{Die Kursivierung im Zitat stammt nicht von Raoul Schrott.}
  \citep[S. 5]{Schrott2016ErsteErde}.
Er vergisst --~oder übergeht~-- zu oft das Wörtchen \glqq auch\grqq.
Das geht so wesentlich gegen das Erhabene der naturwissenschaftlichen
  Erkenntnis, um die es ihm eigentlich bestellt zu sein scheint. 
Besonders die theoretische Physik und Mathematik, aber im Kern jede
  Naturwissenschaft steht aber gerade dazu in Opposition.
Selbstverständlich wird sie den menschlichen Einfluss und von ihrem
  geschichtlichen Verlauf her allein schon in ihrer inspirativen Dimension
  nicht die mythischen Einflüsse los.
Aber gerade die beiden großen Säulen der Physik, die Allgemeine
  Relativitätstheorie, die Grundlage der in der Ersten Erde verhandelten
  physikalischen Kosmologien / -gonien, und die Quantenmechanik, mit der sich
  Schrott nachweislich schon lange beschäftigt (siehe \zB\ sein Gedichtband
  \emph{Tropen}), sind paradigmatisch für wissenschaftliche Einsichten, die
  aus guten physikalischen --~insbesondere empirischen~-- und aber genauso 
  theoretisch-philosophischen Gründen geradezu dadurch ausgezeichnet zu sein
  scheinen, in weiten Teilen von der Alltagserfahrung, dem gesunden
  Menschenverstand, mythischen Vorstellungen und menschlichen Begriffen zu
  differieren.
Ganz viel wissenschaftliche Feinarbeit ist sowohl von den ganz Großen des
  Fachs als auch von einem Meer an Forschern geleistet worden, um nachzuweisen,
  ob sich die Quantenmechanik letztlich doch mit intuitiveren, menschlicheren
  Vorstellungen fassen lasse.
Aber alle scheiterten und wie es (gleichzeitig
  theoretisch wie auch empirisch) aussieht auf eine ganz wesentliche und tiefe Weise
  (vom EPR-Paradoxon über die Bellschen Ungleichungen bis hin zum
  \mbox{Kochen-Specker-Theorem}).
Es ist eher so, dass die Physik versucht, auszuhandeln, wie die Dinge, die sie
  untersucht \emph{wirklich}, \dhei\ undabhängig davon, wie wir sie gerne hätten,
  damit sie leicht eingänglich sind, funktionieren; zumindest, wie wir sie beschreiben können
  und in jedem Fall hat sich herausgestellt, dass sich gerade je weiter die
  Wissenschaft fortschreitet (d.\,h. je mehr Phänomene sie erfolgreich in ihr
  Weltbild bei möglichst wenigen Annahmen eingliedern kann), desto ferner
  sich die Theoriegegenstände vom Menschen und unseren mythischen
  Vorstellungen entfernen zu scheinen.
Niemand weiß, ob sich nicht alles eines Tages in einem einfachen Bild
  zusammenfassen lässt und dieses aus welchen Gründen auch immer eine Nähe zu
  mythologischen Vorstellungen besitzt.
Nach bestem Wissen und Gewissen: Es sieht einfach nicht danach aus.
Genauso selbstverständlich haben wir Menschen dann die Tendenz, uns dennoch
  diese amenschlichen Umstände \glqq mit [unseren] menschlichen Maßstäben
  modellhaft anschaulich [zu] machen.\grqq, was Schrott aber
  selbst-inkonsistent"-erweise als nicht möglich abtut \citep[vgl. S. 5]{Schrott2016ErsteErde}. 
Er demonstriert dem zum Trotz weiter hinten im Text, wie gut dies geht; \ZB\
  als er die Abstands- und Größenverhältnisse im Sonnensystem mit von einem Auto fallendem
  Obst illustriert \citep[S. 48 f.]{Schrott2016ErsteErde}.


Angebrachter wäre, der Mensch ist \emph{ein} Ma\ss\ \emph{vieler} Dinge und für beide
  Ausprägungen (wie stark er Ma\ss\ des jeweiligen Dinges und für welche
  überhaupt) wiederum ein Ma\ss\ anzugeben, anstatt penetrant gerade den
  Errungenschaften des menschlichen Geistes, bei denen er am fairsten von sich
  abstand nimmt --~sich methodisch heraushällt~-- und Strukturen des
  Weltganzen offenlegt, die so ganz und gar nicht anthropomorph sind, zu
  unterstellen, sie seien wesentlich von menschlichen --~vor allem
  mythischen und insbesondere eben~-- fiktiven Gehalten durchsetzt.
Vielleicht ist es eher so, dass es eben eine Mannigfaltigkeit an menschlichen
  Vorstellungen, Geschichten und Mythen gibt, die auf \emph{manche}
  wissenschafltichen Ergebnisse \emph{teilweise} passen; In diesen Fällen
  empfehlen sich diese \glqq Geschichte\grqq\ dann natürlich dem Geist an, da
  sie of grundmenschle Erfahrungen verpacken und so der abstrakte Gehalt
  leichter vermittelt werden kann.
Aber dann kann man freilich auch ein Narrativ spinnen, in dem diese Vorstellungen,
  Geschichten und Mythen unseren heutigen wissenschaftlichen Vorstellungen vermeintlich
  wesentlich vorrausgehen und es so darstellen, als treten wir quasi auf der Stelle und
  seien nicht fähig wesentlich weiter über den Tellerrand hinauszusehen, als
  wir es in Homers Odyssee, Hesiods Theogonie oder Maori-Mythen taten.

\newpage
\section{Zur Literarisierung}

Im zweiten Kapitel, Erstes Licht II, im ersten Buches innerhalb des Epos kann man
  folgende Entdeckung machen:
Die Versanzahl pro Strophe entspricht der Zählung der Abschnitte.
Mit anderen Worten: Die Strophen im $n$-ten Abschnitt des ersten 
  Kapitels haben genau $n$ Verse; Mit jedem Abschnitt kommt zu den Strophen in
  ihm eine Zeile mehr hinzu.
Mit noch anderen Worten: Die Länge der Strophen hängt \emph{linear} von der
  Zahl, die die Anordnung der Abschnitte zählt, ab; Sie sind direkt
  proportional zu einander. 
Bezeichnet man mit $\sharp_{\text{Strophe}}$ die Anzahl der Verse in einer Strophe und
  $N_{\text{Abschnitt}}$, kann man den Zusammenhang auch formal als
  \begin{equation}
    \sharp_{\text{Strophe}}(N_{\text{Abschnitt}}) = 1 \cdot N_{\text{Abschnitt}}
  \end{equation}
  schreiben; die einfachste Form eines linearen Zusammenhangs (nämlich sogar
  der Identität).\footnote{Die eingefügte $1$ hat nur den Zweck, die
  strukturelle Ähnlichkeit dieser und der gleich eingeführten physikalischen
  Gleichung noch augenscheinlicher zu Tage treten zu lassen.
Die runden Klammern symbolisieren den \emph{funktionalen} Zusammenhang: $f(x)
  = y$ bedeutet (und liest sich als) \glqq der Ausdruck $f$ ist eine Funktion
  von $x$ und hat an der Stelle $x$ den Wert $y$; ander: $f$ ordnet dem Element
  $x$ das Elemente $y$ zu. \Dhei\ $f$ hängt von $x$ ab; \glqq ändert seinen
Wert, wenn sich $x$ ändert (in der Regel).}

Der Grund, hier diesen Zusammenhang so penetrant hin zu seinem mathematischen
  Ausdruck zu treiben, ist der Folgende:
Wenn man sich vergegewärtigt, dass hier wissenschaftliche Theorien über die
  Kosmogonie verhandelt werden, dann denkt liegt die Idee nicht fern, dass
  die Ausdehnung des Alls --~oder damit verbunden an das Hubble-Gesetz~-- hier
  strukturell, formal umgesetzt wird (oder zumindest wahrscheinlich in die
  Form hineingelesen werden kann).
Edwin Hubble erkannte in den Spektren entfernter Galaxien (die er für
  \glqq Nebelsterne\grqq\ hielt), dass diese \emph{rotverschoben} sind.
Ein Effekt, der Auftritt, wenn sich der Sender einer Welle (in diesem Fall
  Licht) vom Empfänger während des Ausstrahlens der Welle entfernt.
Dadurch erscheint die Wellenlänge nämlich länger, da sich der Sender zwischen
  dem Aussenden zweier Wellenberge noch entfernt und diese Länge zur
  eigentlichen Entfernung der Wellenberge noch hinzukommt.\footnote{Das
    Adjektiv \glqq rotverschoben\grqq\ kommt daher, dass Rotes Licht eine
    längere Wellenlänge, eben den Abstand der Wellenberge in Licht dieser
    Farbe, hat als alle andere für uns wahrnehmbaren Farben.
  In diesem Sinne nennt man Strahlung längerer Wellenlänge \emph{röter} als
    Strahlung kürzerer Wellenlänge.}

Das Hubble-Gesetz in seiner urspünglichen Form handelt auch von so einer
  direkten Proportionalität; und zwar zwischen Entfernungsgeschwindigkeit
  $v_{\text{Galaxie}}$ von entfernten Galaxien und ihrer Entfernung
  $d_{\text{Galaxie}}$ von uns:
  \begin{equation}
    v_{\text{Galaxie}}(d_{\text{Galaxie}}) = H_{0} \cdot d_{\text{Galaxie}}\text{.}
  \end{equation}

Dieser Zusammanhang gilt für \emph{alle} Galaxien! \Dhei\ je weiter eine jede
  Galaxie von uns entfernt ist, desto schneller bewegt sie sich von uns weg
  \emph{in eben jenem Maße, dass eine $n$-mal soweit entfernte Galaxie sich
  $n$-mal so schnell von uns entfernt}; die Entfernungsgeschwindigkeiten von
  Galaxien verhalten sich strukturell zu ihrer Entfernung zu uns so, wie sich
  die Strophenlängen (Versanzahl) zum Abstand des jeweiligen Abschnitts zum
  Beginn des zweiten Kapitels.

Besonders schön wäre, wenn sich also einerseits zwischen der
  Entfernungsgeschwindigkeit der Galaxie von uns fort und der Strophenlänge
  und andererseits zwischen der Entfernung der Galaxie von uns und der
  Entfernung zum Beginn des Anfangs des zweiten Kapitels Entsprechungen finden
  ließen.
Bei zweiterem Entpreschungspaar ist das offensichtlich; ihr tertium
  comperationis ist ihre metrische Funktion.
  Beide messen einen Abstand: einmal im Raum (\zB\ in der Einheit Metern
  messbar) und das andere Mal im Buch (\zB\ in den Einheiten Buchstaben,
  Wörtern, Seiten messbar).
Bei ersterem --~wichtigerem, da es um dessen funktionales Verhalten zu gehen
  scheint~-- Entsprechungspaar, ist dies aber schwieriger.
Hier hinkt die Analogie, denn unterschiedliche Geschwindigkeiten sind etwas
  anderes als unterschiedliche Strophenlängen.
Nicht zuletzt ist der Grund für ihre wesentliche Unterschiedenheit die Zeit;
  die Geschwindigkeit ist eine eine Dynamik beschreibende Größe, ein Maß dafür,
  wie schnell sich eine statische Größe ändert; die Strophenlänge ist vorerst
  eine statische Eigenschaft einer Texteinheit.
Nun gibt es drei Möglichkeiten:
\begin{enumerate}
  \item Konstatieren, dass die hier vermutlich strukturell umgesetzte Analogie
      zwischen der Expnasion des Universums und der Strophenlängenzunahme zwar
      in Maßen plausibel ist und im Wesentlichen eben durch eine lineare Zunahme
      auf der einen Seite einn lineare Zunahme auf der anderen Seite anklingt.
    Sowohl Eleganz als auch Richtigkeit wären unvollkommen.
  \item Die Beobachtung über den Text irgendwie \glqq dynamisieren\grqq, \dh\
    ihr eine dynamische --~und im besten Fall \glqq
    geschwindigkeitsartige\grqq\ Komponente entlocken.
  \item Die Hypothese verwerfen, auf der Seite der physikalischen Phänomene
    und Gesetze steht nicht das Hubbelsche Gesetz Pate für die Auffälligkeit,
    dass die Strophenlänge mit jedem Abschnitt zunimmt, sondern ein
    statischerer Befund.
\end{enumerate}

Möglichkeit 1 zu wählen, ohne 2 und 3 nicht versucht zu haben, wäre billig und
  denkfaul.

Braucht man für einen ähnlich ausgedehnten Text ähnlich lange, dann ist die
  Konsequenz für das Leseerlebnis, wenn man eine Strophe als Sinneinheit
  versteht, dass die Zeit, die man zum Lesen einer Sinneinheit braucht, mit der
  Entfernung zum Anfang des Epos linear zusammenhängt.
Man kann aus diesem statischen Zusammenhang auf räumlicher Ausdehnungsebene
  des Textes über den Leseprozess, der notwendigerweise auch zeitlich ist, auf
  der Seite der Wahrnehmung des Lesers eine dynamische, zeitliche Beobachtung
  gewinnen.

  Formel

Hinkt auch weil ...

Darf er trotzdem weil ...

Symmetriebrechung ...

Darstellungstheorie ...
