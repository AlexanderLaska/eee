%! TEX root = ../main.tex
%*****************************************
\chapter{\"{A}sthetik und Ethik des Erhabenen}\label{ch:ethikUndAestehtik}
%*****************************************

Lou Andreas-Salom\'e schreibt in ihrem Essay von 1910 \glqq Die Erotik\grqq\
  folgende passende Zeilen:
\begin{quote}
  Überall da nämlich, wo das eigene Material sich ihr [der Betrachtungsweise]
    über Sinne und Verstand hinaus ins Unkontrollierbare entzieht, während sie
    es doch auch noch da als existent in ihrem Sinne feststellen, oder sogar
    noch praktisch einschätzen kann.
  Von jenseits der kurzen Kontrollstrecke, die unsrer Beaufsichtigung allein
    zugänglich ist, ergibt sich für das innerhalb ihrer gelegene ein
    veränderter Maßstab hinsichtlich \frqq Wahrheit\flqq\ und \frqq
    Wirklichkeit\flqq.
  Auch das am stofflichsten Greifbare, auch das logisch
    Begreifbarste wird, daran gemessen, zu einer menschlich
    sanktionierten Konvention, zu einem Wegweiser für praktische
    Orientierungszwecke, -- darüber hinaus sich verflüchtigend in den gleichen
    bloßen Symbolwert, wie das von uns als \frqq geistig\flqq\ oder \frqq
    seelisch\flqq\ Erfasste.
  Und an \emph{beiden}Enden unsres Weges erhebt sich damit so unübertretbar
    das Gebot: \frqq Du \emph{sollst} dir nur in Zeichen und Vergleichen
    Beredte, worauf alle Geistesschilderung angewiesen bleibt, sich mit
    aufgenommen sieht in den Grundwert menschlicher Erkenntnisweise. Wie in
    jenem Horizontstrich, von Schritt zu Schritt vor uns zurückweichen,
    schließt sich dennoch auch immer wieder \flqq Himmer und Erde\flqq\ für
    uns zusammen zu \emph{einem} Bilde: die uranfängliche Augentäuschung, --
    und zugleich das letzte Symbol.
  \citep{Salome1910}.
\end{quote}
Auch Hegel hat hier etwas hinzuzufügen: \glqq Schon unsere äußeren
  Anschauungen, Beobachtungen und Wahrnehmungen sind oft täuschend und irrig,
  aber noch viel mehr sind es die inneren Vorstellungen, wenn sie auch die
  größte Lebendigkeit in sich haben und uns unwiderstehlich zur Leidenschaft
  fortreißen.\grqq\ \citep[S. 41]{Hegel1986}

\newpage
\section{Das Erhabene aus Sicht einflussreicher Denktraditionen im Verlauf der
Geschichte}

Der Begriff des Erhabenen hat eine dichte Geschichte und vielleicht eine
  blühende Zukunft.
In der westlichen Denktradition kann der Begriff des Erhabenen einerseits auf das
  Altgriechische \textgreek{ὕψος} (hypsos), das selbst die Substantivierung des
  Adverbs \textgreek{ὕψι} (hypsi: \emph{empor}, \emph{hoch oben}) --~und damit
  verwandt mit der Präposition \textgreek{ὑπέρ} (hyper, \emph{oberhalb},
  \emph{jenseits}, \emph{sehr})~-- ist und daher zwar in etwa als \emph{Höhe},
  \emph{Gipfel}, \emph{Großartigkeit} aber auch \emph{Stolz} übersetzt wird
  aber eine gewisse \glqq Konnotationenwolke\grqq\ um sich herum besitzt, die
  sich schwer in den Übersetzungen wiederspiegelt.
Andererseits auf das Lateinische \emph{sublimis}, 
%In der westlichen Antike sowohl im 18. Jahrhundert durch ...., dann eingeschlafen und
  %schließlich rehabilitert und ... \citep[vgl. S. 275 f.]{Heininger2010}.


\subsection{Antike}
Der Begriff des Erhabenen wurde zuerst im Zusammenhang mit der Rhetorik
(Pseudo-Longinus) und dann der Poetik (Boileau) thematisch. [ändern!]

\subsection{Burke}

\subsection{Kant}

Immanuel Kannt behandelt das Erhabene im Wesentlichen in zwei Schriften:
  In seinen \emph{Beobachtungen über das Gefühl des Schönen und
  Erhabenen} \citep{Kant1764} und in der \emph{Kritik der Urteilskraft}
  \citep{Kant1790}.
Ersteres lehnt sich noch stark an den englischen Einfluss an und weißt daher
  eher einen phänomenologischen Charakter auf; es ist eher \glqq
  psychologisch-anthropologisch\grqq\ orientiert \citep[S. 125]{Bertinetto2007}.
Letzteres hat sich auf der Folie der beiden anderen Kritiken vom englischen
  Empirismus gelöst und ist zwar immer noch psychologisch aber ansonsten eher
  formalistischer Natur.

Die Empfindung des Schönen wird für Kant durch die \emph{Form} eines
\emph{begrenzten} Gegenstandes in uns \emph{evoziert} \citep[vgl. S. 47 f. u. 48
  ff.]{Kant1790}, die Empfindung des
  Erhabenen hingegen durch die \emph{Darstellung} eines \emph{formlosen},
  \emph{unbestimmten} Gegenstandes \citep[vgl. S. 105 ff.]{Kant1790}.
Als Beispiel für Schönes nennt Kant einfach Blumen, Tiere oder Musik ohne
  Text:
  \begin{quote}
    Blumen sind freie Naturschönheiten. Was eine Bulme für ein Ding sein soll,
    weiß außer dem Botaniker schwerlich sonst jemand, und selbst dieser, der
    daran das Befruchtungsorgan der Pflanze erkennt, nimmt, wenn er darüber
    durch Geschmack urteilt, auf diesen Naturzweck keine Rücksicht. Es wird
    also keine Vollkommenheit von irgend einer Art, keine innere
    Zweckmässigkeit, auf welche sich die Zusammensetzung des Mannigfaltigen
    beziehe, diesem Urteile zum Grunde gelegt. Viele Vögel (der Papagei, der
    Kolibri, der Paradiesvogel), eine Menge Schaltiere des Meeres sind für
    sich Schönheiten, die gar keinem nach Begriffen in Ansehung seines Zwecks
    bestimmten Gegenstande zukommen, sondern frei und für sich gefallen. So
    bedeuten die Zeichnungen \emph{\'a la grecque}, das Laubwerk zu
    Einfassungen oder auf Papiertapeten usw. für sich nichts;
    sie stellen nichts vor, kein Objekt unter einem bestimmten Begriffe,
    und sind freie Schönheiten. Man kann auch das, was man in der Musik
    Phantasieen (ohne Thema) nennt, ja die ganze Musik ohne Text zu derselben
    Art zählen.
  \end{quote} \citep[S. 83 f.]{Kant1790}
Mit der Formlosigkeit geht für ihn auch das Unendliche mit einher \citep[vgl. S. 105]{Kant1790}.
Auffällig ist, dass hier nicht der \emph{Inhalt} sondern die Form im
  Vordergrund steht.
Da das Erhabene aber gerade durch die Formlosigkeit des Gegenstandes charakterisiert
  ist, --~was aber durchaus ein formaler Aspekt ist,~-- tritt an Stelle der
  Form hier die Darstellung des Gegenstandes.
Das ist durchaus eine sehr moderne Haltung, denn so ist für Kant nicht der
  Gegenstand selbst erhaben --~Erhabenheit also keine Objektqualität~--,
  sondern kommt durch das Zusammenspiel von Einbildungskraft und Vernunft
  zustande \citep[vgl. S. ?]{Kant1790}.
So ist also die unmittelbare Empfindung, die ausgegelöst wird, die der
  Erhabenheit und der Gegenstand, der sie evoziert, nur nomenklatorisch \glqq
  erhaben\grqq.
Beim Schönen tritt unmittelbar ein Gefühl ein, dass eher postiv ist: Kant
  spricht vom Gefühl der \glqq Beförderung der Lebenskräfte\grqq.
Auf der Seite des Erhabenen wird nur indirekt --~dann aber viel heftiger~--
  diese Beförderung der Lebenkräfte erregt, da vorerst eine Hemmung im
  Zusammenspiel von Wahrnehmung und Vernunft stattfindet, da dieser
  Verarbeitungsapparat von der Formlosigkeit, Unbestimmt oder eben
  Unendlichkeit irgendwie überfordert ist.
Erst im Überwinden diese überfordert Seins tritt die Einsicht ein,
  dass wir vernunftbegabt und dadurch auch über solche Phänomene, erhaben
  sind.
Kurz: Das Schöne rührt von qualitativen Aspekten der Form eines bestimmten
  Gegenstandes her und das Erhabene von quantitativen Aspekten der Darstellung
  eines unbestimmten Gegenstandes.

Wenn Kant seine Unterscheidung des Schönen vom Erhabenen also auf der formalen
  Ebene der Objektseite tätigt, so arbeiten \emph{topologische} Kategorien und
  Einsichten im Hintergrund.
Das~Schöne teile viele Eigenschaften mit dem Erhabenen, aber unterscheide sich
  eben wesentlich von ihm dadurch, dass das Schöne \emph{irgendwie} \glqq
  begrenzt\grqq\ sei, wobei das Erhabene Formlosigkeit aufweise, \dhei
  --~in Kants eigenen Worten~-- \glqq unbegrenzt\grqq\ sei.
Damit habe für Kant das Erhabene wesentlich mit dem Unendlichen zu tun.
Wichtig ist hierbei die Kursivierung des Wörtchens \glqq irgendwie\grqq, denn
  Kant spart aus, \emph{was} genau \emph{in welchem Sinne} unbegrenzt ist.
In Ermangelung dieser beiden Angaben können Gedankengänge, die topologisch
  geleitet sind, leicht ins Leere laufen.
Ein ganz einfaches Parade(gegen)beispiel aus der Differentialgeometrie, die
  auf der mathematischen Topologie aufbaut, ist eine zweidimensionale
  Kugeloberfläche.
Sie hat keinen Rand und ist so als Fläche nicht begrenzt.
(Sie selbst ist natürlich als der Rand, \dhei\ die Grenze, einer Kugel, die sie
  umgäbe, auffassbar; so wie Erdkruste den restlichen Erdball umgibt.
Sie selbst für sich genommen, \dhei\ als zweidimensionales Gebildes --~also
  wie der Film einer Seifenblase, der auch keine Seifenkugel umgibt~-- besitzt
  keinen Rand, ist aber \emph{endlich}.)
Gleichzeit könnte man nach einer anderen Eigenschaft des Objekts fragen: Nicht
  die Fläche in den Fokus stellen, sondern die Punkte. Der Anzahl ist nähmlich
  bei der endlichen Fläche durchaus unendlich. So gesehen tritt doch eine
  Unendlichkeit auf.
Exakt das lässt sich sehr präzise abstrahieren: Bezüglich einer unvorstellbar
  --~in diesem Fall wirklich unendlich~-- großen Menge an Eigenschaften, \dhei\
  in den verschiedensten abstrakten Räumen mit den verschiedensten
  Dimensionen.
Genau das ist das Feld der mathematischen Topologie.
Sie fasst Aussagen über Benachbarungsverhältnisse glasklar begrifflich formal
  so, dass Aussagen über Begrenzungen (Ränder), innen Liegendes (Inhalte),
  Unendlichkeiten, Annäherung (Konvergenzen), Entfernen (Divergenz) etc. hart
  beweis- und auch wiederlegbar werden.
Sie versieht eine beliebige Menge, von einer ganz konkreten Gruppierung
  klassischer Gegenstände wie endlich vieler Äpfel in einem Korb bis hin zu
  beliebig abstrakten Gruppierungen wie hochdimensionale geometrische Räume
  oder gar so etwas wie die Menge politischer Einstellungen aller Menschen) 
  mit einer gewissen Struktur $\mathcal{O}$: einer Topologie, die festlegt,
  was sogenannte \glqq offene Mengen\grqq\ sind.
Mithilfe derer kann dann klar definieren, was Begriffe wie \glqq innen\grqq,
  \glqq außen\grqq\, \glqq in der Umgebung von\grqq\ etc. bedeuten und so auch
  Aussagen wie, dass aus \glqq Formlosigkeit ein Bezug Unendlichkeit\grqq\
  bestünde, lass sich dann im konkreten, \dhei\ wohldefinierten Rahmen, klären
  und im Allgemeinen ist es völlig offen, ob diese beiden Eigenschaften etwas
  miteinander zu tun haben.

Jedenfalls unterteilt Kant das Erhabene noch weiter in das sogenannte \glqq
  Mathematisch-Erhabene\grqq\ und das \glqq Dynamisch-Erhabene\grqq.
Im Ersteren Fall, ist die reine Quantität so groß, dass die Einbildungskraft
  nicht stark genug ist. Das Phänomen zeigt sich über alle (menschlichen) Maßen
  groß, lang, tief, weit entfernt, winzig, bunt, vielzählig etc. und die
  Einbildungskraft es nicht vermag, eine konsistente Form des Ganzen zu
  manifestieren.
Auf dieses erste Moment folgt das zweite einer Leistung der Vernunft, da wir
  Ideen begreifen können, die mit dem Unendlichen, nicht Bestimmten umgehen
  können: paradigmatisch natürlich die des Unendlichen in der Mathematik
  (daher vermutlich der Name).
Im zweiten Fall ist des Menschen Urteilkraft nicht der reinen Anzahl vorerst
  unterlegen, sondern es stellt sich bei ihm ein Erahnen \glqq unsere[r]
  physische[n] Ohnmacht\grqq\ \citep[S. 261 f. ??]{Kant1790} ein.
Wiederum wird dieses erst hemmende Moment von einem zweiten überhoben, denn es
  führt dem in diesem Fall das Erhabene wahrnehmenden Menschen vor, dass es in
  uns ein Vermögen gibt, dass über diesen Naturgewaltet steht.
Wir bleiben trotz der dynamischen Kräfte gewaltvoller Naturprozesse, denen wir
  unterliegen auf anderer Ebene \glqq in unserer Person unerniedrigt\grqq.
Ethik und Ästhetik? ...

Zusammengefasst: Es ist keineswegs gesagt, dass Formlosigkeit, \dhei\ schwierige
  Be-/Abgrenzbarkeit mit Unendlichkeit und damit immenser Größe einhergehe.
Ganz im Gegenteil, es lassen sich Eigenschaften von Objekten in einem ganz
  allgemeinen Sinn finden, die gleichzeitig nicht begrenzt und dennoch endlich
  (wie eben beispielsweise eine Kugeloberfläche) sind.
Darüberhinaus ist die Analyse ganz sensibel davon abhängig, auf welche Weise
  man beurteilt, was begrenzt ist und welche Eigenschaft man sich ansieht und
  mit welchem Maß man misst, was \glqq Form\grqq\ und damit \glqq
  Formlosigkeit\grqq\ überhaupt im jeweilig gegeben Rahmen ist. (Technisch
  entspricht dies der Wahl der Topolgie auf der Menge, um die es geht; also
  der Wahl, welche Teilmengen der Menge man als offen/geschlossen auffassen
  \emph{will}.)

Je gewaltiger, größer, unermesslicher ein Gegenstand, desto erhabener wird er
  --~für Kant~-- empfunden.
Vielleicht wäre eine Gesamtansicht der Erde oder Ansichten großer Ausschnitte
  des Universums mit all ihren Milliarden Galaxien das Erhabenste für ihn
  gewesen, wäre ihm diese Darstellungen schon zugänglich gewesen.
Solche unermesslichen Objekte, Gegenstände, Sachverhalte darzustellen, ist der
  Anspruch und das Verdienst eines Projektes wie der \emph{Ersten Erde}.
Alleine die Darstellung des Konzepts unseres Sonnensystems ist wohl \glqq
  unermäßlich\grqq\ für uns.
In der Passage ..., in der er die Abstände der Planeten untereinander und zur
  Sonne und die Größen der Planeten an sich mit von einem Auto fallenden
  Obststücken vergleicht, leistet er genau diese Überwindungsarbeit der
  Hemmung durch \glqq schlechtin Großes\grqq, die so wesentlich für die
  Empfindung des Erhabenen zu sein scheint, indem er durch einen geschickt
  gewählten Vergleich die Ermessenhürde zumindest passabel überspringt.

Sogleich empfiehlt sich aber die Frage an, inwieweit die Einbettung dieses
  bildlichen Vergleichs mit Alltagsgegenständen einen Mehrwert in Form der
  Einbettung dessen in seine lyrische Form und insbesondere als Teil eines
  Epos produziert.
Was ist die besondere Leistung des Vergleichs, den er nicht schon bar seiner
  epischen Einbettung besäße?

\subsection{Schiller und Goethe}
\subsection{Hegel}
\graffito{Diese Gedanken beruhen auf Mitschriften Hegels berliner~Vorlesungen der Jahre 
  1820 bis 1829 von seinen Studenten und sind im Rahmen des \glqq Systems der Wissenschaften\grqq\
  ausgehend von~seiner \glqq Phänomenologie des Geistes\grqq\ \citep{Hegel1987} gedacht.} 
Hegel schlie\ss t in seinen \emph{Vorlesungen über die Ästhetik}
  \citep{Hegel1986} schon von Beginn an aus seiner \glqq Philosophie der schönen Kunst\grqq,
   wie die Ästhetik --~ginge es nach ihm~-- eigentlich benannt sein sollte, \glqq
  sogleich das \emph{Naturschöne}\grqq\ aus \citep[S. 13]{Hegel1986} und begr\"{u}ndet
  diese Entscheidung damit, dass \zB\ die Sonne \glqq als ein \emph{absolut
  notwendiges} Moment\grqq\ nicht geistiger Natur ist; nicht \glqq in sich
  frei und selbstbewu\ss t\grqq\ \citep[S. 14]{Hegel1986}.
Dar\"{u}berhinaus böte das Naturschöne zu wenig Angriffsfläche für
  philosophisches Interesse, da wir \glqq uns zu sehr im \emph{Unbestimmten},
  ohne \emph{Kriterium} zu sein, [fühlen]\grqq\ \citep[S.15]{Hegel1986}.
Es fehlen also in dieser Domäne des Seienden Ma\ss stäbe, die einen Vergleich
  und damit eine genaue Untersuchung bewerkstelligen würden (vgl. hierzu die ersten
  Seiten der Einleitung zur \emph{Phänomenologie des Geistes} 
  \citep[S. ?? f.]{Hegel1987}).
Hierbei handelt es sich um ein generelles Phänomen, dass die theoretische
  Physik und Mathematik in ganz anderem --~aber nicht minder fundamentalem~-- Rahmen
  unfassbar gewinnbringend --~zumindest teilweise~-- sehr erfolgreich gelöst zu
  haben scheinen und \ua\ auf die Konzepte \emph{Yang-Mills-Theorie},
  \emph{Eichfeldtheorien}, \emph{Faserbündel} und nicht zuletzt \emph{Zusammenhang} führten.
Denn manchmal ist es für die wesentlichen Aussagen einer theoretischen Untersuchung gänzlich
  irrelevant, wie man den Vergleich genau bewerkstelligt, nur dass man es tut
  und dies auf eine konsistente wohl formalisierte Weise, und es ist daher möglich 
  in der Theoriegenese zu berücksichtigen, dass die Theorie, um zu
  funktionieren, mehr \glqq Ma\ss stabsbalast\grqq\ benötigt, als wirklich in
  der Natur realisiert ist.
\graffito{All die hervorgehobenen Theoriebausteine finden weiter unten noch genauer
  Erwähnung und sollen hier nur aufgrund des passenden Zusammenhangs schon
  einmal Erwähnung finden.}
Man muss nur \zB\ an jedem Punkt in Raum und Zeit etwas vorgeben, dies verwenden 
  und kann es am Ende effektlos wieder
  herausrechnen oder das Ergebins ist schon per constructionem der Theorie gar
  nicht von gewissen Annahmen abhängig.
Hegel ist die Problematik offenbar bewusst, kommt aber nicht zu einer solchen
  Lösung --, was ihm sicherlich nicht anzukreiden ist, da es vielleicht nicht
  übertrieben ist, den ganzen Ideenkreis der \emph{Kovarianz},
  \emph{Eichfreiheit} und eben der oben erwähnten theoretischen
  Begriffsbildungen eine der grö\ss ten Errungenschaften des menschlichen
  Geistes der letzten Jahrhunderte zu nennen~--, denn er schlägt bewusst einen
  anderen Weg ein (vgl. hierzu das \glqq Lehmrutenbeispiel\grqq\ aus der
  Einleitung der \emph{Phänomenologie des Geistes} \citep{S.~??}{Hegel1987}
  oder ebd. auch das Beispiel des sich im Wasser brechenden Lichtstrahls,
  bei denen für Hegel nach dem Abziehen der theoretischen 
  intervenierenden Artefakte \emph{nichts} bleibt und dies nicht als eine 
  konstruktiv konnotierte \emph{bestimmte Negation} (...).
Er findet gerade aber über dieses Konzept und seine damit einhergehende dreischrittige 
(These, Antithese und Synthese) aber eher 
  sich immerfort spiralisch windende Denkfigur (These, Antithese, die die These
  bestimmt negiert und dann der Synthese, die wieder These und Antithese
  bestimmt negiert und aber selbst sogleich den Status einer neuen These
  besitzt, die wieder Ausgangspunkt für eine weitere Windung ist) der
  \emph{Dialektik}.
Hegels Ästhetik hier in den Diskurs einzubinden, kann dennoch von Vorteil sein, denn
  einerseits ist das hier betrachtete Erhabene nicht nur das Erhabene des
  Natuschönen, sondern gerade entschieden das Erhabene der theoretischen
  Einsichten in die Struktur des Naturschönen. Es kommt so also das, was Hegel
  ausschlie\ss t mit dem Kern dessen, was ihn interessiert und er einschlie\ss
  t zusammen: \glqq die \emph{aus dem Geiste geborene und wiedergeborene} Schönheit\grqq\ 
  \citep[S.14]{Hegel1986}.
Sie kann also, wenn es um das Naturschöne, das in der \emph{Ersten Erde}
  besungen wird, kontrastisch ex negativo und wenn es um die Theorie geht, die
  genauso besungen werden und die ja geistige Produkte par excellence sind,
  direkt und für sich genommen, Anwendung finden.
Darüberhinaus scheint Hegels ästhetische Denkbewegung eine Einsicht
  zu unterliegen, die die Einfachheit einer Binsenwei\ss heit besitzt und die
  aber einem von der Schönheit einer Theorie überzeugten Physiker zur Wei\ss
  glut zu bringen vermag:
  \glqq[\ldots] wenn die Kunst gerade die lichtlose \emph{dürre Trockenheit}
  (Diese Hervorhebung stammt nicht von Hegel.) des Begriffs erheiternd belebe,
  seine Abstraktionen und Entzweiun mit der Wirklich versöhne, den Begriff an
  der Wirklichkeit ergänze, [\ldots so] beschäftigt sich die Wissenschaft 
  [\emph{i}]\emph{hrem Inhalte nach} mit dem in sich selbst
  \emph{Notwendigen}.\grqq\ \citep[S.~19]{Hegel1986}.
Die hoffnungsvolle Idee der Versöhnung der Kunst und der Wissenschaft ist
  nicht das, was hier kritisiert werden soll; dies ist gerade das Wunderschöne an
  Schrotts Epos; er realisiert hier gerade das, was an dieser Stelle Hegel und an anderer
  der Physiker Richard Feynmann (speziell im Rahmen der Poesie und der
  Naturwissenschaften) einfordern (...).
Vielmehr ist es die Unterstellung der dürren Trockenheit der Begriffsbildungen
  der Wissenschaften, ihrer Begriffsbildungen und Erkenntnisse, die hinkt.
Wieder kann sich also ex negativo und \glqq ex positivo\grqq\ im Abarbeiten an
  Hegel etwas Gewinnbringedes beim Herauspreparieren des besonderen
  Status des Schönen und Erhabenen bei der Berührung von Kunst und
  Wissenschaft im Allgemeinen und Literatur und Physik im Konkreten ergeben. 
